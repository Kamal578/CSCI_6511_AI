\documentclass[11pt]{article}

\usepackage[a4paper,margin=1in]{geometry}
\usepackage{amsmath}
\usepackage{amssymb}
\usepackage{graphicx}
\usepackage{hyperref}
\usepackage{listings}
\usepackage{float}

\hypersetup{
    colorlinks=true,
    linkcolor=blue,
    urlcolor=blue
}

\title{
    \textbf{CSCI 6511 – Artificial Intelligence}\\
    \vspace{0.3cm}
    \textbf{Project 1: Sub Project C – N-Puzzle}\\
    \vspace{0.3cm}
    A* Search Solution
}

\author{
    Ahmadov Kamal \\
    Imamverdiyev Omar
}

\date{\today}

\begin{document}
\maketitle

% --------------------------------------------------
\section{Introduction}

The N-Puzzle is a classic search problem in artificial intelligence that involves rearranging numbered tiles on an $n \times n$ grid into a predefined goal configuration using the minimum number of moves. One tile position is blank, and legal actions consist of sliding an adjacent tile horizontally or vertically into the blank space.

This project implements a solution to the N-Puzzle for grid sizes $3 \leq n \leq 8$ using the A* search algorithm. The objective is to find an optimal (minimum-length) sequence of moves that transforms a given initial configuration into the goal state.

% --------------------------------------------------
\section{Problem Description}

An N-Puzzle instance consists of:
\begin{itemize}
    \item An $n \times n$ grid containing tiles numbered from $1$ to $n^2 - 1$
    \item One blank space represented by the value $0$
    \item Legal moves that slide a tile into the blank position
\end{itemize}

The goal state is defined as:
\[
(1, 2, 3, \dots, n^2 - 1, 0)
\]
with the blank tile located in the bottom-right corner.

The task is to find the shortest sequence of moves (up, down, left, right) that reaches the goal configuration.

% --------------------------------------------------
\section{Algorithm Choice: A* Search}

To solve the N-Puzzle efficiently and optimally, we use the A* search algorithm. A* is an informed search method that expands nodes based on the evaluation function:
\[
f(n) = g(n) + h(n)
\]
where:
\begin{itemize}
    \item $g(n)$ is the cost from the start state to the current state
    \item $h(n)$ is a heuristic estimate of the remaining cost to reach the goal
\end{itemize}

A* guarantees optimality provided that the heuristic function is admissible (never overestimates the true cost).

\subsection{Heuristic Function}

Our implementation uses a combination of two admissible heuristics:
\begin{itemize}
    \item \textbf{Manhattan Distance}: the sum of the vertical and horizontal distances of each tile from its goal position.
    \item \textbf{Linear Conflict}: an enhancement that adds extra cost when two tiles are in the same row or column as their goal positions but in reversed order.
\end{itemize}

The final heuristic is:
\[
h(n) = h_{\text{Manhattan}}(n) + h_{\text{Linear Conflict}}(n)
\]

This heuristic remains admissible and significantly improves search performance.

% --------------------------------------------------
\section{Implementation Details}

The program is implemented in Python and follows a command-line interface design.

Key components include:
\begin{itemize}
    \item Robust input parsing supporting tab-delimited and space-aligned formats
    \item Solvability checking using inversion count rules
    \item A priority queue (min-heap) for A* frontier management
    \item State expansion with cost tracking
    \item Parent pointers for path reconstruction
\end{itemize}

A closed-set mechanism ensures that previously explored states with higher cost are not revisited.

No graphical user interface is used, in accordance with assignment guidelines.

% --------------------------------------------------
\section{How to Run the Program}

The program is executed from the command line using Python.

\begin{verbatim}
python npuzzle_astar.py input.txt
python npuzzle_astar.py input.txt --show
python npuzzle_astar.py input.txt --evaluation
python npuzzle_astar.py input.txt --evaluation --show
\end{verbatim}

\begin{itemize}
    \item \texttt{input.txt} contains the initial puzzle configuration
    \item The optional \texttt{--show} flag prints all intermediate board states along the solution path
    \item The optional \texttt{--evaluation} flag enables performance evaluation mode, in which Uniform Cost Search (h = 0) and A* search with heuristics are executed and compared
\end{itemize}

When both \texttt{--evaluation} and \texttt{--show} are specified, the program prints the full solution path \emph{only} for the heuristic-guided A* search. The solution path for Uniform Cost Search is intentionally omitted, as it may involve a large number of intermediate states and does not provide additional insight beyond the reported performance metrics.



% --------------------------------------------------
\section{Experimental Results}

For solvable configurations, the implemented algorithm produces the following outputs:
\begin{itemize}
    \item The minimum number of moves required to reach the goal state
    \item The corresponding sequence of moves using the symbols U, D, L, and R
    \item Optionally, the complete sequence of intermediate board configurations
\end{itemize}

For unsolvable configurations, the program correctly detects the impossibility of reaching the goal state and reports that no solution exists.

\subsection{Instructor-Provided 5$\times$5 Puzzle}

To validate the correctness and robustness of the implementation on larger problem instances, an instructor-provided 5$\times$5 N-Puzzle configuration was tested. The initial state of the puzzle is shown below:

\begin{verbatim}
11  1  2  3 14
12  7  9 10 13
 6  8 18  5  4
21 16 17 19 15
22 23    24 20
\end{verbatim}

This configuration was verified to be solvable using the parity-based solvability test. The algorithm successfully found an optimal solution for this instance.

\subsection{Solution Output}

The solver reported the following results for the given 5$\times$5 puzzle:

\begin{itemize}
    \item \textbf{Minimum number of moves:} 38
    \item \textbf{Move sequence:}  
    \texttt{LLURRULLUURRRDDRUULDDRUULDLLDLURRDRRDD}
\end{itemize}

The obtained solution demonstrates the ability of the A* algorithm with admissible heuristics to handle larger puzzle sizes beyond the standard 4$\times$4 case. Despite the increased state space of the 5$\times$5 puzzle, the algorithm efficiently guided the search toward the goal state and produced a valid optimal solution.

These experimental results confirm both the correctness of the implementation and its practical applicability to instructor-provided benchmark inputs.


% --------------------------------------------------
\section{Evaluation and Heuristic Comparison}

To evaluate the impact of heuristic guidance in A* search, we compare the performance of A* with and without heuristics on a representative 4$\times$4 (15-puzzle) instance. The goal of this experiment is to demonstrate how heuristics influence search efficiency, memory usage, and runtime, while maintaining solution optimality.

\subsection{Compared Methods}

The following two approaches are evaluated:

\begin{itemize}
    \item \textbf{Uniform Cost Search (UCS):}  
    Implemented as A* search with heuristic function $h(n) = 0$. In this case, state expansion is based solely on path cost, resulting in an uninformed search strategy.

    \item \textbf{A* with Heuristic:}  
    A* search using an admissible heuristic composed of Manhattan distance combined with the linear conflict heuristic.
\end{itemize}

Both methods guarantee optimal solutions.

\subsection{Evaluation Metrics}

The comparison is based on the following metrics:

\begin{itemize}
    \item Number of expanded states
    \item Maximum frontier size
    \item Runtime
    \item Optimal solution length
\end{itemize}

All experiments were conducted on the same initial configuration to ensure fairness.

\subsection{Experimental Results}

The evaluated 4$\times$4 puzzle required an optimal solution of 12 moves. Table~\ref{tab:quantitative_comparison} presents the measured performance of both search strategies.

\begin{table}[H]
\centering
\begin{tabular}{|l|c|c|}
\hline
\textbf{Metric} & \textbf{UCS (h = 0)} & \textbf{A* with Heuristic} \\
\hline
Expanded states & 8,562 & 14 \\
\hline
Maximum frontier size & 7,909 & 13 \\
\hline
Runtime (seconds) & 0.072 & $<$ 0.001 \\
\hline
Solution length & 12 & 12 \\
\hline
\end{tabular}
\caption{Quantitative comparison of UCS and A* on a 4$\times$4 puzzle}
\label{tab:quantitative_comparison}
\end{table}

\subsection{Discussion}

Both Uniform Cost Search and A* search successfully produced the same optimal solution of length 12, confirming the correctness and fairness of the comparison. However, the computational effort required by the two methods differed substantially.

Uniform Cost Search expanded 8,562 states and maintained a frontier of up to 7,909 states, reflecting the exponential growth typical of uninformed search strategies. In contrast, A* search with heuristic guidance expanded only 14 states and required a maximum frontier size of 13 states. This corresponds to a reduction of more than two orders of magnitude in the explored search space.

The runtime measurements further illustrate this difference. While UCS required measurable execution time, the heuristic-guided A* search completed almost instantaneously. These results clearly demonstrate that admissible heuristics dramatically improve search efficiency while preserving optimality, even for moderate-sized N-Puzzle instances.



% --------------------------------------------------
\section{Limitations and Discussion}

The state space of the N-Puzzle grows exponentially with grid size. While A* with admissible heuristics performs efficiently for small and medium puzzles (e.g., 3×3 and 4×4), larger instances (6×6 to 8×8) may require significant memory and computation time.

Despite these limitations, A* remains an appropriate and optimal choice for this problem within the given constraints.

% --------------------------------------------------
\section{Conclusion}

This project demonstrates the application of A* search to solve the N-Puzzle optimally. By combining Manhattan distance with linear conflict heuristics, the implementation efficiently finds shortest solutions while adhering strictly to the project requirements. The solution illustrates key concepts of informed search and heuristic design in artificial intelligence.

\end{document}
